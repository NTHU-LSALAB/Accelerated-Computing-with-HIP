\titleformat{\subsection}{\large\bfseries\CJKsection}{\thesubsection}{1em}{}

\chapter{ \term{HIP} 程式設計入門} \label{chap:getting_started_with_hip_programming}
\section{介紹}

在介紹章節中,我們強調了 \term{HIP} 是在 \term{ROCm} 平台上發揮 AMD GPUs 平行處理能力的首選語言。對於熟悉 \term{C/C++} 語法的開發者而言, \term{HIP} 可以被視作 \term{C/C++} 的擴充,加入了額外的語法和函式庫接口(例如 GPU 程式設計 API )。雖然掌握這些 API 起初可能會有些挑戰,但經過一定的練習後,語法將變得相對容易理解。一個學習 \term{HIP} 的實用方法是通過查看範例程式的語法並追蹤其運作流程。在本章中,我們將介紹兩個基礎的 \term{HIP} 程式:HelloWorld 和 VectorAdd,以幫助讀者理解 \term{HIP} 的語法、程式架構和執行流程。在接下來的章節中,我們將介紹更進階的功能。

\section{初次見面, \term{HIP} 世界 -- Hello World 範例}

在許多人在學習程式設計的旅程中,通常會從實作各種程式語言的「Hello World」程式開始。秉持這個傳統,本書將使用 GPU 來表達這個問候。這個應用程式會觸發一個 GPU thread 來印出「Hello World」的訊息。

要在 GPU 上啟動任務,程式設計者必須首先編寫針對 GPU 執行的程式碼。我們的範例如Listing 2.1 所示。kernel函式 \term{gpuHello} 的功能非常簡單,就是印出「Hello World」。 GPU kernel 本質上是一個回傳類型為 \code{void} 的函數(即沒有回傳資料)。要將一個函數變成一個 kernel ,我們必須在函數的定義前加上前綴 \code{\_\_global\_\_},以讓編譯器生成針對 GPU 的特定程式碼。

除了 GPU kernel 外,程式設計者還需要撰寫程式的 CPU 部分(即 host program),如 main 函數所示。 GPU 無法獨立運作,需要與 CPU 密切協作。要啟動 GPU kernel (即 GPU kernel launch),我們需要像呼叫普通 CPU 函數一樣呼叫 GPU kernel 函數。但不同之處在於,函數名稱與參數列表之間需要插入 \code{<<<>>>} 符號,該符號用於指定 kernel 中的 threads 數量。在此範例中,我們使用 1, 1 表示我們只希望建立一個 thread 。至於使用兩個數字的原因,我們將在後續進一步說明。

\begin{lstlisting}[language=C, caption={「Hello World」\term{HIP}程式範例}, label={lst:example}]
#include <hip/hip_runtime.h>

__global__ void gpuHello() {
    printf("Hello World\n");
}

int main() {
    gpuHello<<<1, 1>>>();
    hipDeviceSynchronize();
}
\end{lstlisting}

要編譯這個程式,\term{ROCm} 平台提供了一個基於 Clang 的 \term{HIP} 編譯器 -- \term{hipcc}。在安裝了 \term{ROCm} 的平台上,可以透過命令列介面(CLI)使用 \bold{hipcc} 來呼叫編譯器。附錄 A 詳細說明了如何安裝 \term{hipcc} 以及其他工具和套件。要編譯我們的 \term{helloWorld.cpp} 原始檔,可以使用指令 \code{hpicc helloWorld.cpp -o helloWorld} 。編譯完成後,與一般的 CPU 程式相似,只需輸入 \code{./helloWorld} 即可執行程式

\section{使用 \term{HIP} 處理資料 -- Vector Add 範例}

HelloWorld 程式可能是你接觸 GPU 程式設計的第一步,但它的實用性相當有限。GPU 的設計目的是為了高吞吐量的資料處理。因此,在接下來的範例中,我們將示範如何使用 GPU 進行兩個向量的元素加法運算,並將結果儲存到第三個向量中。

\subsection{平行運算的可能性}

眾所周知, GPU 在資料處理上運行快速,因為它們可以平行處理這些資料。因此,使用 GPU 進行資料處理的第一步是分析平行運算的潛在可能性。要分析 Vector Add (向量加法)範例中平行運算的可能性,讓我們嘗試先使用 CPU 來實作該程式,如 Listing 2.2 所示。
\begin{lstlisting}[language=C, caption={\term{HIP} 錯誤偵測範例}, label={2nd:example}]
int main() {
    int n; // 陣列中的元素個數
    float *a, *b, *c;

    // 配置並初始化 a、b、c

    for(int i = 0; i < n; ++i) {
        c[i] = a[i] + b[i];
    }

    // 使用計算結果並釋放記憶體資源

    return 0;
}
\end{lstlisting}

實作的核心是一個 for 迴圈。平行運算的潛力來自傳統上會被建立在迴圈中的程式碼,因為需要多次重複執行某些操作。在這個例子中,每次迴圈迭代處理向量中的一個位置。由於迴圈中的每次迭代彼此獨立,因此他們可以以任何順序執行。
這裡我們可以確定這些迭代相互獨立,因此可以被平行處理。這個問題原始的特性使其非常適合使用 GPU 來解決。

在本節的其餘部分,我們將介紹幾個概念,再實作 GPU 程式。我們將著重於如何管理 threads 和記憶體。

\subsection{管理 threads}
在啟動 GPU kernel 時,會創建許多執行相同 kernel 函數的 threads 。一個 kernel 生成的所有 threads 集合被稱為 grid 。

以 VectorAdd 範例為例,一維 grid 可能是最適合該程式的選擇。然而,某些應用可能需要處理本質上是二維(例如矩陣)或三維(例如空間)資料的情況。\term{HIP} 提供了創建二維和三維 grid 的能力。不過,目前我們將專注於一維 grid (網格)。

\term{HIP} 更將 threads 組織為 blocks (區塊),每個 block 通常包含 32 至 1024 個 threads 。 block 內的 threads 彼此之間可以互動和同步。本範例並不要求 threads 之間的同步和溝通,但我們會在更進階的範例中探索這些功能的使用方式。

當使用 \code{<<<>>>} 符號啟動 kernel 時,需要提供兩個數字,分別表示 grid 大小( block 數量)和 block 大小(即每個 block 中的 thread 數量)。總 thread 數量是 grid 大小與 block 大小的乘積。需要注意的是, \term{HIP} 不允許創建部分 block。

在編寫 kernel 程式碼之前,我們必須先理解 \term{HIP} 中 thread 的組織方式。在 CPU 上,使用一個 thread 處理一筆資料是可行的,但管理 CPU thread 的巨大開銷,可能會快速地降低性能。幸運的是, GPU 的 threads 所造成的開銷較小,因此通常會使用一個 thread 處理一筆資料。

接下來,讓我們先編寫 GPU thread 的程式碼(參見Listing 2.3)

\begin{lstlisting}[language=C, caption={Vector\_add GPU kernel}, label={3rd:example}]
// HIP kernel: 每個 thread 負責一個 c 的元素
__global__ void vecAdd(
    double *a, double *b, double *c,
    int n
) {
    // 取得全域的 thread 編號 (thread id)
    int id = blockIdx.x * blockDim.x + threadIdx.x;

    // 確保我們不會越界存取
    if(id < n) {
        c[id] = a[id] = b[id];
    }
}
\end{lstlisting}

kernel 函數(kernel function)是為個別 thread 的執行而設計的。在程式碼中, \code{int id = blockIdx.x * blockDim.x + threadIdx.x}; 生成了每個 thread 在整個 grid 中的唯一標識符 id。這個標識符對於區分每個 thread 非常重要,確保它們能處理不同的資料元素或執行特定的操作。

該計算依賴於幾個內建的變數來計算全域唯一的標識符。以下是這些內建變數的含義:
\begin{itemize}
    \item \code{blockIdx.x}: 表示當前 block 在 grid 中 x 軸方向的索引值。由於 block 是在 grid 中被安排的,每個 block 在 grid 中都會有唯一的索引值。
    \item \code{blockDim.x}: 表示每個 block 中的 x 軸方向 threads 的數量。
    \item \code{threadIdx.x}: 表示當前 thread 在 block 中 x 軸方向的索引值。
\end{itemize}

在這裡,後綴 \code{.x} 表示 x 軸上的索引,而 \code{.y} 和 \code{.z} 則分別用於二維和三維kernel。

在 multi-block (多區塊)的情況下,表達式 \code{blockIdx.x * blockDim.x} 計算當前 block 中第一個 thread 的索引值。這個計算很重要,因為每個 block 可能包含多個 threads ,因此該表達式有效地計算了當前 block 中第一個 thread 相對於整個 grid 中所有 threads 的「偏移量」。

將這個偏移量加上 \code{threadIdx.x},就能生成整個grid中當前thread的全域索引值(global index),即 \code{id}。這個全域索引對每個 thread 都是唯一的,使每個 thread 能夠知道自己在 grid 中的位置,並據此執行操作,例如處理一個唯一的資料元素。

接著進行加法操作(add action)。首先,通過 \code{if} 條件語句執行邊界檢查。由於無法只啟動部分 blocks ,而資料可能無法與 blocks 完美對齊,因此可能存在某些 threads 不需要執行任何操作的情況。例如,如果每個陣列中只有 40 個數字,我們仍然必須啟動 64 個 threads ,因為只能啟動完整的 block 。在這種情況下,最後的 24 個 threads 將沒有資料可處理。

\begin{lstlisting}[language=C, caption={管理 threads 與啟動 kernel}, label={4th:example}]
// 一個 block 中 threads 的數量
blockSize = 64;

// 一個 grid 中 blocks 的數量
gridSize = (int)ceil((float) n / blockSize);

// 啟動 kernel
vecAdd<<<gridSize, blockSize>>>(GPUArrayA, GPUArrayB, GPUArrayC, n);
\end{lstlisting}

最終在第 11 行執行加法操作,就和 CPU 的實作類似。為了啟動 kernel ,需要指定 block 大小(block size)和 grid 大小(grid size)。我們將 block 大小設為 64(選擇此數字的原因將在 第 8.1 節 中討論)。理想情況下,總 thread 數應等於 n(資料元素數量)。然而,考慮到 block 大小的限制,我們需要將thread數量向上取整到最近的block邊界。因此才使用以下程式碼啟動內核。

需要注意的是,我們目前還沒有 \code{GPUArrayA}、\code{GPUArrayB} 和 \code{GPUArrayC},因為資料仍然在 CPU 端。我們將在下一步介紹如何準備這些資料。

\subsection{使用 \term{HIP} API 進行資料搬移}
在任何程式執行的開始,我們假設資料由 CPU 進行維護。通常來說,如果需要處理大型資料集,該資料集不是從檔案中載入,就是從網際網路下載,而這些操作是 GPU 無法直接執行的。在這個簡單的範例中,我們使用 \code{Malloc} 函數來分配一些 CPU 資料,並隨機生成資料(請參見 Listing 2.5)。

\begin{lstlisting}[language=C, caption={CPU的記憶體配置}, label={5th:example}]
// 在這裡宣告所有的 CPU 陣列
// 每個向量 (vector) 的大小 (bytes)
size_t bytes = n*sizeof(double);

// 在 host 端為每個向量配置記憶體
CPUArrayA = (double*) malloc(bytes);
CPUArrayB = (double*) malloc(bytes);
CPUArrayC = (double*) malloc(bytes);

// 在 host 端初始化向量
for(int i = 0; i < n; i++){
    CPUArrayA[i] = i;
    CPUArrayB[i] = i;
}
\end{lstlisting}

資料位於 CPU 端,我們需要將資料從 CPU 複製到 GPU,以便 GPU 可以處理。Kernel 執行完後,我們也需要將資料從 GPU 複製回 CPU 以使用結果。此外,我們還需要在 GPU 上分配記憶體空間,作為接收資料的緩衝區。為了實現這些目標,\term{HIP} 提供了一組記憶體管理的 API,如下所列。

\begin{itemize}
    \item \term{hipMalloc}:此 API 用於分配 GPU 記憶體,其功能類似於 CPU 的 \term{malloc()} 函數。此 API 分配一塊指定大小的記憶體,用於在 GPU 上保存 \term{HIP} 所指定的資料結構。需要注意的是,分配的記憶體大小受限於 GPU 的物理記憶體容量,不同 GPU 之間可能有所差異。\term{hipMalloc} 的語法如下:\code{hipMalloc(void **ptr, size\_t size)},其中 \code{**ptr} 是指向我們的資料結構的指標,\code{size} 指定記憶體分配的大小。例如,若我們希望在 GPU 上分配一個單精度浮點數陣列 \code{dev\_A},容量為 1,024 個元素,可使用以下語法:\code{hipMalloc((void**)\&dev\_A, 1024*sizeof(float)))}。需要注意的是,此為阻塞式 API 呼叫 (blocking API call),即記憶體分配完成之前下一行程式碼不會被執行。
    \item \term{hipMemcpy}:此 API 用於在 CPU 和 GPU 之間傳輸資料,反之亦然。第一個參數是目標陣列,第二個參數是來源陣列,第三個參數指定要傳輸的資料大小,第四個參數則定義傳輸的方向。在進行傳輸時,需確保資料大小與目標緩衝區相符,且資料來源區域的範圍有效,否則可能會導致錯誤或隨機值。\code{hipMemcpy} 的語法如下:
    
    \code{hipMemcpy(void *dst, const void *src, size\_t sizeBytes, hipMemcpyKind kind)}。其中,\code{*dst} 是目標緩衝區指標,\code{*src} 是來源緩衝區指標,\code{sizeBytes} 是傳遞資料的大小,第四個參數指定傳輸方向,可選值包括:(1) \code{hipMemcpyHostToDevice} 將資料從CPU傳遞到GPU、(2) \code{hipMemcpyDeviceToHost} 將資料從GPU傳遞到CPU、(3) \code{hipMemcpyHostToHost} 將資料從CPU傳遞到CPU、(4) \code{hipMemcpyDeviceToDevice} 將資料從GPU傳遞到GPU。舉例來說,若我們有一個GPU緩衝區\code{dev\_A}存放浮點數,一個CPU緩衝區\code{host\_A},且希望將 CPU 緩衝區 \code{host\_A} 中的 1,024 個浮點數值複製到 GPU 緩衝區 \code{dev\_A} 中,我們的hipMemcpy呼叫語法如下:\code{hipMemcpy(dev\_A, host\_A, 1024*sizeof(float), hipMemcpyHostToDevice)}。同樣地,這也是一個阻塞式 API 呼叫。
    \item \term{hipFree}:與 \term{hipMalloc} 分配記憶體相對,當不再需要使用分配的 GPU 記憶體時,應該釋放記憶體。這可以通過 \term{hipFree} API 實現,其語法為:\code{hipFree(void *ptr)}。舉例來說,若想釋放最初通過 \term{hipMalloc} 分配的一個變數 float *dev\_A,則可以使用 \code{hipFree(dev\_A)}。
\end{itemize}

Listing 2.6 中的程式碼,展示了如何在 VectorAdd 範例中使用這些 \term{HIP} API。具體來說,我們分別為向量 A、B 和 C 使用 \code{hipMalloc} API 分配了三個緩衝區。然後,我們將輸入資料(向量 A 和 B)複製到 GPU 端。在 kernel 執行後,我們將輸出向量 C 從 GPU 端複製回 CPU 端。最後,我們釋放分配的記憶體。

\begin{lstlisting}[language=C, caption={VectorAdd 範例中的 GPU 資料管理}, label={6th:example}]
// 在這裡宣告所有的 GPU 陣列
hipMalloc(&GPUArrayA, bytes));
hipMalloc(&GPUArrayB, bytes));
hipMalloc(&GPUArrayC, bytes));

// 將 host 向量資料複製到 device
hipMemcpy(GPUArrayA, CPUArrayA, bytes, hipMemcpyHostToDevice));
hipMemcpy(GPUArrayB, CPUArrayB, bytes, hipMemcpyHostToDevice));

// 啟動 kernel 相關的程式碼

// 將資料從 device 複製回 host
hipMemcpy(CPUArrayC, GPUArrayC, bytes, hipMemcpyDeviceToHost));

// 使用資料

// 釋放 device 記憶體
hipFree(GPUArrayA);
hipFree(GPUArrayB);
hipFree(GPUArrayC);
\end{lstlisting}

\subsection{錯誤與正確性}
在使用 \term{HIP} API 時,確保每次呼叫成功執行是很重要的,無論是記憶體分配還是資料傳輸。一種常見的方法是建立一個 assertion macro(例如 如 Listing 2.7 所示的\bold{HIP\_ASSERT})。在範例中,我們可以看到 \term{hipMalloc} 呼叫被該 assertion 包裝。如果呼叫失敗,程式會立即在此處終止。此外,如果未對每個 runtime API 呼叫進行 assertion 檢查,程式可能會在後續執行(例如 kernel 啟動)時失敗,而問題來源可能無法立即明確。例如,如果程式設計者試圖將 6 GB 的資料分配到僅有 4 GB 總記憶體的 GPU 中,若未檢查 API 呼叫結果,呼叫可能看起來會是成功的。接下來啟動 kernel 時,程式可能會崩潰或產生錯誤的結果。因此,在每次 API 呼叫後使用 assertion 是一種良好的程式設計習慣。

\begin{lstlisting}[language=C, caption={\term{HIP} 錯誤檢查範例}, label={7th:example}]
#define HIP_ASSERT(x)(assert((x)==hipSuccess))
#define NUM1024
int main() {
    float* gpuA = 0;
    HIP_ASSERT(hipMalloc((void**)&gpuA, NUM * sizeof(float)));
}
\end{lstlisting}

\subsection{組裝統整}

我們在 Listing 2.8 中列出了完整的 VectorAdd 範例程式碼。我們之前已介紹了程式碼的大部分內容。需要特別指出的是第 62 行呼叫 \code{hipDeviceSynchronize} 。這行程式碼被安插在 kernel 啟動與裝置到 device 到 host 的記憶體複製操作之後。由於 kernel 啟動是一個非同步的過程,此函數呼叫是必要的。 kernel 呼叫會在 GPU 上的執行尚未完成時便會返回。如果沒有這個明確的同步,在正確結果計算完成前就會將 device 的記憶體複製到 host ,導致複製到錯誤的結果。

\begin{lstlisting}[language=C, caption={完整的VectorAdd範例程式碼}, label={8th:example}]
#include "hip/hip_runtime.h"
#include <stdio.h>
#include <stdlib.h>
#include <math.h>

#define \term{HIP}_ASSERT(x)(assert((x)==hipSuccess))

// HIP kernel。每個 thread 負責 c 的一個元素
__global__ void vecAdd(double *a,double *b, double *c, int n){
    // 獲得我們的全域thread ID
    int id =blockIdx.x*blockDim.x+threadIdx.x;

    // 確保我們不會越界存取
     if (id < n){
        c[id] = a[id] = b[id];
     }
}

int main(int argc, char* argc[]) {
    // 根據元素數量定向量的大小,以byte為單位
    int n = 10240;
    size_t bytes = n * sizeof(double);

    // Host 向量
    double *CPUArrayA;
    double *CPUArrayB;
    double *CPUArrayC;
    double *CPUVerifyArrayC;

    // Device 向量
    double *GPUArrayA;
    double *GPUArrayB;
    double *GPUArrayC;

    // 為每個 host 上的向量分配記憶體
    CPUArrayA =(double*)malloc(bytes);
    CPUArrayB =(double*)malloc(bytes);
    CPUArrayC =(double*)malloc(bytes);
    CPUVerifyArrayC =(double*)malloc(bytes);

    // 為每個 GPU 上的向量分配記憶體
    \term{HIP}_ASSERT(hipMalloc(&GPUArrayA, bytes));
    \term{HIP}_ASSERT(hipMalloc(&GPUArrayB, bytes));
    \term{HIP}_ASSERT(hipMalloc(&GPUArrayC, bytes));

    // 初始化 host 上的向量
    for(int i = 0; i < n; i++) {
        CPUArrayA[i] = i;
        CPUArrayB[i] = i + 1;
    }

    // 將每個 host 上的向量複製到 device
    \term{HIP}_ASSERT(hipMemcpy(GPUArrayA, CPUArrayA, bytes, hipMemcpyHostToDevice));
    \term{HIP}_ASSERT(hipMemcpy(GPUArrayB, CPUArrayB, bytes, hipMemcpyHostToDevice));

    // 計算 grid 與 block 的大小
    int blockSize = 256;
    int gridSize = (int) ceil((float) n / blockSize);

    // 執行 kernel
    vecAdd<<<gridSize,blockSize>>>(GPUArrayA,GPUArrayB,GPUArrayC,n);
    hipDeviceSynchronize();

    // 將陣列資料複製回 host
    \term{HIP}_ASSERT(hipMemcpy(CPUArrayC,GPUArrayC, bytes, hipMemcpyDeviceToHost));

    // 在 CPU 上計算結果
    for(int i = 0; i < n; i++){
        CPUVerifyArrayC[i]= CPUArrayA[i]+ CPUArrayB[i];
    }

    // 驗證正確性
    for(int i = 0; i < n; i++){
        if (abs(CPUVerifyArrayC[i]-CPUArrayC[i]) > 1e-5){
            printf("Error atposition i %d,Expected: %f, Found: %f \n", i, CPUVerifyArrayC[i], CPUArrayC[i]);
        }
    }

    // 釋放 device 記憶體
    \term{HIP}_ASSERT(hipFree(&GPUArrayA, bytes));
    \term{HIP}_ASSERT(hipFree(&GPUArrayB, bytes));
    \term{HIP}_ASSERT(hipFreec(&GPUArrayC, bytes));

    // 釋放 host 記憶體
    free(CPUArrayA);
    free(CPUArrayB);
    free(CPUArrayC);
    free(CPUVerifyArrayC);

    return 0;
}
\end{lstlisting}

\subsection{結語}
在本章中,我們學習了如何使用 \term{HIP} 在 GPU 上進行平行程式設計的基礎知識。從非常基本的“Hello World”範例開始,我們逐步進行到在 GPU 上的向量加法操作。我們研究了 \term{HIP} 程式的基本結構,這將幫助我們理解本書後續更複雜的範例。此外,我們還介紹了基本的 API,例如分配記憶體、在裝置之間搬移資料以及 kernel 啟動的相關操作。最後,我們瞭解到在 API 可能呼叫失敗情況下使用檢查和assertion,以及透過標準 CPU 實現進行正確性驗證的重要性。

作為總結,讓我們回顧 GPU 程式執行中常見的步驟。這些步驟如下:

\begin{itemize}
    \item 分配所有資料結構的 CPU 記憶體緩衝區,並填充為程式所需的值。一般來說有兩種類型的資料結構:一種用於提供給 GPU 的輸入,另一種用於存放計算後 GPU 輸出的複製資料。
    \item 使用 \term{HIP Runtime} API (如 \term{hipMalloc})為 GPU 記憶體緩衝區分配類似的資料結構。
    \item 在分配完成後,使用 \term{hipMemcpy} runtime API 將資料從 CPU 緩衝區複製到 GPU 緩衝區。這確保在啟動 kernel 之前,已填充到 CPU 緩衝區的資料也存在於 GPU 緩衝區中。
    \item 設定 GPU kernel 的 grid 和 block 大小。儘管可以通過 kernel 啟動函式完成此操作,但為了程式碼的清晰性,建議將此步驟分開。這一步提供了很大的靈活性,允許我們選擇最有效的 block 大小。我們將在後續章節中討論此問題。
    \item 在 GPU 上啟動 kernel。
    \item 使用 \term{hipDeviceSynchronize} 同步化 CPU 和 GPU 。由於我們尚未進行除錯,且所有的指令都在預設的 stream 被處理,這步驟目前並非必要。
    \item 將所需資料從 GPU 複製回 CPU。這一步必須完成以獲取 GPU kernel 的結果(例如,將其存儲到檔案中、顯示到標準輸出,或在 CPU 上執行任何後續處理步驟)。此時檢查程式碼的正確性是很重要的,正如下一節所述。
    \item 從 GPU 收集資料值後,使用 hipFree API 釋放在 GPU 上分配的所有記憶體。
\end{itemize}

在接下來的章節中,我們將介紹更進階的範例,主要討論 \term{HIP} 支援的額外功能。